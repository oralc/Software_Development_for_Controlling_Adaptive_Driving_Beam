\chapter{References/Bibliography}
\label{sec: Bibliography}


Below you will find two sample references sections (bibliographies) covering the main types of publications (Internet source, standard, monograph, article in a book, technical rule, article in a specialist journal, AI tool). Information on other publication types can be found in ONR~12658 (for German) and in ISO~690 (for English), both of which are cited below. ONR~12658 and ISO~690 are available in the library of the Wels Campus and online, respectively.

\setlength{\parindent}{0mm}
\section{Bibliography for literature references within the text (AGR, AT, MB, PDK) or within a footnote (AMM, MEWI)}

\printbibliography[heading=none,env=bibliographyAlpha]
\nocite{*}

\vspace{1mm}
\fbox{\parbox{\textwidth}{In bibliographic software, predefined citation styles similar to this representation can be found via keywords such as “ISO 690”. 
Using such citation styles may necessitate manual modifications before you submit your thesis.}}


\section{Bibliography for literature references within the text (BI, BUT, IPM, LCW, LTE)}

Austrian Standards Institute (2013) \textit{ONR 12658:2013 Empfehlungen zum Zitieren von Informationsquellen und Anleitungen zur Gestaltung von Literatur- und anderen Quellennachweisen in wissenschaftlichen Arbeiten.}

\vspace{1mm}
Austrian Standards International (2023) \textit{ÖNORM A 2662:2023 Wissenschaftliche Abschlussarbeiten – Angaben für den bibliographischen Nachweis.}

\vspace{1mm}
Beuermann, C. (2013) \textit{Die Entdeckung des menschlichen Einflusses auf das Klima.} Available at: 
\texttt{https://www.bpb.de/gesellschaft/umwelt/klimawandel/38444/}
\texttt{entdeckung-des-menschlichen-einflusses} (accessed: 14 July 2021).

\vspace{1mm}
International Organization for Standardization (2021) \textit{ISO 690:2021 Information and documentation – Guidelines for bibliographic references and citations to information resources}.

\vspace{1mm}
Karmasin, M. und Ribing, R. (2019) \textit{Die Gestaltung wissenschaftlicher Arbeiten: Ein Leitfaden für Facharbeit/VWA, Seminararbeiten, Bachelor-, Master-, Magister- und Diplomarbeiten sowie Dissertationen.}
10\textsuperscript{th} edn. Wien: facultas.

\vspace{1mm}
Lehrndorfer, A. and Reuther, U. (2008) ‘Kontrollierte Sprache – standardisierte Sprache?{’}, in Muthig, J. (ed.) \textit{Standardisierungsmethoden für die Technische Dokumentation.} Lübeck: Schmidt-Römhild, pp. 97 121.

\vspace{1mm}
Meier, J. (2011) \textit{Globalisierung}. Wiesbaden: Pons.

\vspace{1mm}
Müller, E. and Meier, J. (2019) \textit{Globalisierung neu gedacht}. Wiesbaden: Pons.

\vspace{1mm}
Müller, E., Meier, J. and Huber, A. (2021) \textit{Globalisierung: Rahmenbedingungen, Prozesse, Institutionen}. Wiesbaden: Pons.

\vspace{1mm}
Müller, E., Meier, J., Huber, A. and Tausch, G. (2016) \textit{Globalisierung: Gegenwart und Zukunft}. Wiesbaden: Pons.

\vspace{1mm}
OpenAI (ed.) (2023) ChatGPT, version xy [language model]. \textit{Response to the author’s prompt “xyz”}, generated 1 December 2023. Available at: \newline 
\texttt{https://chat.openai.com/} (accessed: 1 December 2023).

\vspace{1mm}
Schulz, M. (2014) ‘Doku-Norm in der Praxis’, \textit{technische kommunikation}, 36(6), pp. 46-49.

\vspace{1mm}
\fbox{\parbox{\textwidth}{In bibliographic software, predefined citation styles similar to this representation can be found via keywords such as “cite them right”. 
Using such citation styles may necessitate manual modifications before you submit your thesis.}}
--

\section{Bibliography for numeric references in square brackets (AB, AET, AGR, AMM, AT, BUT, EE, LTE, MB, MEWI, RSE, SES, VTP, WFT)}
\printbibliography[heading=none]

\vspace{1mm}
\fbox{\parbox{\textwidth}{In bibliographic software, predefined citation styles similar to this representation can be found via keywords such as “ISO 690”. 
Using such citation styles may necessitate manual modifications before you submit your thesis.}}

\newpage
\section*{Appearance}
%The appearance of the bibliography varies depending on the reference style that you must apply in your thesis [see section \ref{sec: TypeOfCitation} on page \pageref{sec: TypeOfCitation}].

\section*{Division}
The list of references can be sub-divided into the following sub-sections, if necessary:
\begin{itemize}
	\item	Primary literature 
	\item	Secondary literature
	\item	Tertiary literature
\end{itemize}

\section*{Literature search}
On the Internet pages of the library of the Wels Campus, you can find a very detailed list with numerous links to:
\begin{itemize}
	\item	electronic journals
	\item	databases
	\item	numerous libraries
	\item	catalogues of the book trade
	\item	patent agencies

\end{itemize}
The library staff are happy to help you with your literature search.

